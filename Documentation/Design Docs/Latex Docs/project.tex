% !TEX root = DesignDocument.tex


\chapter{Project Management}

\section{Team Member's Roles}
Kendra Deziel was in charge of GUI design, implementation, and debugging. All members of the team created wireframe ideas which she concatenated with the intention of keeping to the generic video game interface standards most users will know and understand.
Andrew Stelter was in charge of backend design, implementation, and debugging. This included things like file loading, unit tests, and the design of interfaces between all modules of the project. He was also the scrum master and architect of the project layers.
Ryley Sutton was in charge of the world view design, implementation, and debugging. This is the section in the center of the project where the simulation, map, and robot can be viewed. She selected Qt graphics framework which supports interaction between 2D objects as well as allows for easy rotating. This was chosen over OpenGL because it was readily available, developed to be used in Qt by Qt easily and most importantly was better documented for use with Qt for easier learning.  
Samuel Williams was in charge of the physics design, implementation, and debugging. He selected Box 2D as the physics engine for the project to handle movements and collisions of all world view objects: robot or map object.


\section{Project  Management Approach}
For this project, an adapted agile approach was taken. Our sprints were two weeks long, which we judged to be the best length based on our schedules as students and on the complexity of the project overall. Team meetings were scheduled once weekly, sometimes with bonus 'coding sessions', and client meetings were also scheduled once weekly. 

Trello was used to keep track of backlog items, which were derived from user stories during group meetings. Git was used for our repository.

Communications between group members happened almost exclusively over text and email, or in person.

Backlog items were assigned to specific group members during group meetings after being assigned an adequately high priority.


\section{ Stakeholder Information}
This project's successful completion would not only be a positive occurence for our clients, it could also come to benefit future robotics classes, not only at the South Dakota School of Mines and Technology, but the nation or maybe even the world over.

Anybody currently using a worse alternative for 2D robotics simulation would likely be happy to hear of the successful completion of this project.

\subsection{Customer or End User (Product Owner)}
Dr. Jeff McGough and Christopher Smith are our clients or 'customers'. Their input generates user stories and helps to determine the priority of backlog items at weekly meetings, where they deliver this input to the entire development team in a progress report style dialogue.

\subsection{Developers --Testers}
Andrew Stelter is acting as our lead developer and project manager. Samuel Williams is a developer primarily focused on the physics engine and backend. Kendra Deziel is a developer primarily focused on our GUI. Ryley Sutton is a devloper primarily focused on graphics.

\section{Intellectual Property and Licensing}
This project is going to open source on GitHub. We have yet to select a license. 

\section{Sprint  Overview}
At the beginning of each sprint, a group meeting is used to reevaluate our backlog. This consists of moving items that have been completed to a board specifically for completed items, adding new backlog items generated from user stories, reevaluating the priority of items that are not in progress, and assigning those items of highest priority to specific team members, or in some cases, groups of team members to be completed in a timely manner.

Over the course of the next week, individual work is done by each group member on the items assigned to them with collaboration in the case that one team member needs help from another.

A client meeting occurs, a progress report is given, new user stories are collected, and another group meeting happens to discuss progress amongst ourselves, and to assign additional work in the case that someone has already finished their work for the sprint.

Over the next week, more individual work and testing takes place, there is another client meeting, and completed items are pushed to the repository. Any unfinished items are carried over into the next sprint.

\section{Terminology and Acronyms}
Provide a list of terms used in the document that warrant definition.  Consider 
industry or domain specific terms and acronyms as well as system specific. 

\subsection{ROS - Robot Operating System}
A flexible framework for writing robot software. It is a collection of tools, libraries, and conventions that aim to simplify the task of creating complex and robust robot behavior across a wide variety of robotic platforms.

\subsection{STDR - Simple Two Dimensional Robot Simulator}
A simple robot simulator, and a predecessor to this project. The goal of this project from the start was to improve on the STDR, and although we decided to write our own program from scratch, the STDR has often been used for reference and comparison.

\subsection{SDSMT - South Dakota School of Mines and Technology}
The university at which this project was developed. All authors are students of SDSMT (sometimes SDSM&T).

\section{Sprint Schedule}
The sprint schedule.  Can be tables or graphs.   This can be a list of dates with the visual 
representation given below.

\section{Timeline}
Gantt chart or other type of visual representation of the project timeline.

\section{Development Environment}
The basic purpose for this section is to give a developer all of the necessary 
information to setup their development environment to run, test, and/or develop. 


\section{Development IDE and Tools}
Describe which IDE and provide links to installs and/or reference material. 

\section{Source  Control}
Which source control system is/was used?  How was it setup?  How does a developer 
connect to it? 

\section{Dependencies}
Describe all dependencies associated with developing the system. 

\section{Build  Environment}
How are the packages built?  Are there build scripts? 

\section{Development Machine Setup}
If warranted, provide a list of steps and details associated with setting up a 
machine for use by a developer. 


