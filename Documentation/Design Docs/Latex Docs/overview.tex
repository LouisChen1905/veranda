% !TEX root = DesignDocument.tex


\chapter{Overview, Description and Deliverables}

\section{Team Members and Team Name}
The team, named Team Alpha Robots, consists of the following members:
\begin{itemize}
	\item Kendra Deziel
	\item Ryley Sutton
	\item Samuel Williams
	\item Andrew Stelter
\end{itemize}

\section{Client}
The clients (and sponsors), Dr. McGough and Christopher Smith, would like an improved version of the 2D ROS simulation environment known as the STDR. The client would like an improved physics simulation, as well as increased ease of use so that the simulator can be used as a fast, easy introduction to programming with ROS.

\section{Project}
The project will consist of one or more packages which run in the ROS 2 (Robot Operating System 2) environment. The software will have a graphical interface, both for setting up the system and displaying the simulation. The simulator should allow for importing and exporting robots and map layouts, as well as simulation of said robots and maps in a real-time ROS system. It should be possible to change the connections to the rest of ROS 2 in order to allow external code to control the simulated robots. The simulator should, in turn, produce sensor data which can be sent back to the control code.

The project is to represent a two-dimensional, top-down simulation. It is intended for mobile robots which can be easily represented on a plane. It is not intended for three-dimensional simulation, or side-view simulation (such as with an industrial robotic arm)

The project will run in Ubuntu 16.04, Windows 10, and possibly OSX Sierra or Capitan and may be used as part of a ROS system containing robot control code.

\subsection{Purpose of the System}
The purpose of the system is to provide a simple environment in which ROS-based robot control software can be tested. Current environments (such as Gazebo) add many layers of complexity and can be overwhelming to users new to ROS.

\section{Deliverables}

\subsection{Software}
Source code of the project and all required files to build it as a set of ROS 2 packages. The project should be unit-tested as well as possible, and require minimal setup to use. 

\subsection{Documentation}
Documentation of the following items:
\begin{itemize}
	\item Original requirements for the project
	\item Design of the software
	\item Implementation details of the software
	\item Requirements to build and run the software
	\item Use of the software (User Manual)
	\item Testing and Verification of software correctness
	\item Extension of the software to provide new functionality
\end{itemize}
