% !TEX root = DesignDocument.tex

\chapter{User Stories,  Requirements, and Product Backlog}
\section{Overview}


The overview should take the form of an executive summary.  Give the reader a feel 
for the purpose of the document, what is contained in the document, and an idea 
of the purpose for the system or product. 

 The user stories 
are provided by the stakeholders.  You will create he backlogs and the requirements, and document here.  
This chapter should contain 
details about each of the requirements and how the requirements are or will be 
satisfied in the design and implementation of the system.

Below:   list, describe, and define the requirements in this chapter.  
There could be any number of sub-sections to help provide the necessary level of 
detail. 




\section{User Stories}
\subsection{User Story \#1}
Client wants to have a 2D simulator for robots designed using ROS.

\subsubsection{User Story \#1 Breakdown}

\subsection{User Story \#2} 
Client wants to be able to easily swap control channels between the simulator and an actual robot

\subsubsection{User Story \#2 Breakdown}

\subsection{User Story \#3} 
Client wants to be able to design a robot with some shape for it's body, a drive system, and any number of sensors

\subsubsection{User Story \#3 Breakdown}

\subsection{User Story \#4} 
Client wants to be able to send joystick messages to the control code from the simulator UI

\subsubsection{User Story \#4 Breakdown}

\subsection{User Story \#5}
Client wants to be able to choose what drive system is used (ackermann, differential, mecanum...) and where it is positioned under the robot

\subsubsection{User Story \#5 Breakdown}

\subsection{User Story \#6} 
Client wants to be able to choose what sensors are used (lidar, sonar, touch...) and where they are positioned on top of the robot

\subsubsection{User Story \#6 Breakdown}

\subsection{User Story \#7} 
Client wants to be able to see a visualization of what sensors detect (when applicable)

\subsubsection{User Story \#7 Breakdown}

\subsection{User Story \#8} 
Client wants the system unit tested

\subsubsection{User Story \#8 Breakdown}

\subsection{User Story \#9} 
Client wants to be able to import an image as the map to drive on

\subsubsection{User Story \#9 Breakdown}

\subsection{User Story \#10} 
Client wants to be able to simulate multiple robots

\subsubsection{User Story \#10 Breakdown}

\subsection{User Story \#11} 
Client wants the robot file format to be compatible with Gazebo 3D simulator

\subsubsection{User Story \#11 Breakdown}

\subsection{User Story \#12} 
Client wants the software to be easy to modify for use outside of ROS

\subsubsection{User Story \#12 Breakdown}

\section{Requirements and Design Constraints}
\subsection{System  Requirements}
Software will run as a collection of nodes in ROS Kinetic. This means that it must follow ROS package conventions and be compiled through catkin build system on Ubuntu 16.04.

\subsection{Project  Management Methodology}
The project will be managed using the Agile methodology. Due to the fact that all members of the team are full-time college students, there will only be group meetings twice a week, and at least one of them will be at a time that the Client can attend. Sprints will be two weeks long, beginning and ending on Thursday.

The project will reside in a private GitHub repository; this repo will contain both the project source code and documentation files.

The backlog will be maintained via a Trello board. All tickets will start in the ``Todo: Future" category and will be moved into the ``Todo: Current Sprint" category at the start of the sprint during which they are to be completed. Tickets in progress should be moved to the ``In Progress" category, and completed work should be moved to the ``Code Review" category. Code will be reviewed during weekly meetings, without the client. After the code associated with a ticket is reviewed and approved, the ticket will be moved to the ``Done" category.

\section{Product Backlog}
The full initial product backlog should go here.  The sprint backlogs are located in the prototypes chapter.

 
\begin{itemize}
\item What system will be used to keep track of the backlogs and sprint status?
\item Will all parties have access to the Sprint and Product Backlogs?
\item How many Sprints will encompass this particular project?
\item How long are the Sprint Cycles?
\item Are there restrictions on source control? 
\end{itemize}


\section{Research or Proof of Concept Results}
\subsection{Initial Research}
Samuel Williams and Ryley Sutton researched various 2D physics simulation libraries as well as the possibility of writing one from scratch. They advised the team to use the open-source physics engine ``Box2D"

\subsection{Proof of Concept}
Andrew Stelter constructed a number of projects utilizing ROS alongside various Qt applications. The goal of these projects was to find any issues the catkin build system might have with Qt, and determine the best way to build a project using both of these libraries, which require separate event loops. He advised the team to use Qt 5.5, as it is part of the default Ubuntu 16.04 distribution, and to use Qt Widgets for the UI instead of QML because he was unable to find a way to successfully build and run QML applications on a fresh Ubuntu system with out installing Qt separate from ROS.

\section{Supporting Material}

